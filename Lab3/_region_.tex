\message{ !name(austin_report.tex)}\documentclass[twocolumn]{article}

\begin{document}

\message{ !name(austin_report.tex) !offset(-3) }

\section{The Assignment}

\subsection{Time Dependent PDEs}

In this section, we begin exploring time-dependent partial
differential equations, specifically building up to the
advection-diffusion equation by considering each case individually
first, then combining them into one equation. We will also test
different explicit and implicit finite differencing schemes to step
forward in time and test the Courant-Fiedrichs-Lewy (CFL) condition,
which is usually used to determine the size of the time step.

\subsubsection{The Advection Equation}

The one-dimensional advection equation is a function of
concentration $u(t, x)$ over time:
\begin{equation}
  u_t = c u_x
\end{equation}
where $c$ is the speed of flow. In a similar vein to d'Alembert's
solution to the wave equation, the exact solution to the 1D advection
equation is found by shifting the initial conditions, $u(0, x)$, along
the $x$-axis:
\begin{equation}
  u(t, x) = u(0, x + ct)
\end{equation}
This means that if $c$ is positive, the initial function is shifted to
the left. The opposite is true if $c$ is negative, the initial
function is shifted to the right.

To find a numerical solution to the advection equation, we applied
four different finite difference schemes: forward and backward Euler,
Lax-Wendroff, and Crank-Nicholson. These four schemes will be used on
four different initial conditions, shown in table \ref{table:cases}
with seven different sets of parameters, shown in table
\ref{table:advection}.

\begin{tabular}{c|c}
  Case & Function \\ \hline \hline
  0    & $u(0, x) = \sin(2 x)$ \\ \hline
  1    & $u(0, x) = \begin{cases} 0 & \text{if} \; x < 0 \\ 1
    \text{if} \; x \geq 0 \end{cases}$ \\ \hline
  2    & $u(0, x) = \frac{\delta_{x, 0}}{\Delta x}$ \\ \hline
  3    & $e^{- 4 x^2}$ \\ \hline
  \label{table:cases}
\end{tabular}

\begin{tabular}{c|c|c|c|c}
  Trial & $c$ & $\Delta x$ & $\Delta t$ & $r = c \Delta t / \Delta x$ \\ \hline
  1     & 0.5 & 0.04       & 0.02       & 0.25                        \\ \hline
  2     & 0.5 & 0.02       & 0.02       & 0.5                         \\ \hline
  3     & 0.5 & 0.0137     & 0.02       & 0.728                       \\ \hline
  4     & 0.5 & 0.0101     & 0.02       & 0.99                        \\ \hline
  5     & 0.5 & 0.0099     & 0.02       & 1.11                        \\ \hline
  6     & 0.5 & 0.02       & 0.01       & 0.25                        \\ \hline
  7     & 0.5 & 0.02       & 0.04       & 1                           \\ \hline
  \label{table:advection}
\end{tabular}

%%%%%%%%%%%%%%%%%%%%%% 
% PUT IN THE RESULTS %
%%%%%%%%%%%%%%%%%%%%%% 

\subsubsection{The Diffusion Equation}

The pure diffusion equation describes an initial pulse which diffuses into
the surrounding fluid. The one dimensional diffusion equation is:
\begin{equation}
  u_t = \beta u_{xx}
\end{equation}
where $\beta$ is the diffusion coefficient. The exact solution for $x$ over the
interval $\[0, 1\]$ for the initial condition $u(0, x) = \sin(\pi x)$ and
boundary conditions $u(t, 0) = u(t, 1) = 0$ is:
\begin{equation}
  \label{eq:diffusion}
  u(t, x) = e^{- \beta \pi^2 t} \sin(\pi x)
\end{equation}
In order to solve the equation numerically, we can change the diffusion equation
into a FD equation. Replacing the time derivative with a first order forward
difference scheme and the spatial derivative by a centred difference scheme
results in the equation:
\begin{equation}
  u_i^{n+1} = u_i^n + r \left( u_{i+1}^n - 2 u_i^n + u_{i-1}^n \right)
\end{equation}
where the Courant number is $r = \frac{\beta \Delta t}{(\Delta x)^2$,
  $n$ is along the time axis and $j$ is along the $x$-axis.
The conditions for stability can be found through von Neumann
stability analysis. Starting from the finite difference equation,
stepping forward in time and centred stepping in space:
\begin{equation}
  \frac{u_j^{n+1} - u_j^n}{\Delta t} = \beta \frac{u_{j+1}^n - 2 u_j^n
    + u_{j-1}^n}{\Delta x^2}
\end{equation}
we first assume that the each step can be represented as a product of
an amplification factor $G^n$ and an exponential term so that $u_j^n =
G^n e^{i k j \Delta x}$. This form of $u_j^n$ is then substituted into
the FD equation and solved for $G^1$:
\begin{equation}
  \begin{aligned}
    e^{i k j \Delta x} \frac{G^{n+1} - G^n}{\Delta t} &= \beta G^n
    \frac{e^{i k (j + 1) \Delta x} - 2 e^{i k j \Delta x} + e^{i k (j
        - 1) \Delta x}}{\Delta x^2} \\
    G^1 - 1 &= \frac{\beta \Delta t}{\Delta x^2} \left(e^{- i k \Delta
        x} - 2 + e^{+ i k \Delta x} \right) \\
    \therefore G^1 &= \left( 1 - 2 r \right) + 2 r \cos(k \Delta x)
  \end{aligned}
\end{equation}
where $r = \frac{\beta \Delta t}{\Delta x^2}$ in the last
substitution.

For stability, we impose the condition that $|G^n| \leq 1$. If the
cosine term is equal to one ($\cos(k \Delta x) = 1$), then $|G^n| =
1$, which satisfies the stability condition. However, if the cosine
term is equal to negative one, we use the fact that $r > 0$, so the
stability condition becomes:
\begin{equation}
  \begin{aligned}
    |G^n| &\leq 1 - 4 r \\
    - 1 &\leq 1 - 4 r \\
    \therefore \Delta t &\leq \Delta t_{max} = \frac{\Delta x^2}{2 \beta}
  \end{aligned}
\end{equation}

When the diffusion equation was solved numerically for the conditions
described in equation \ref{eq:diffusion}, it was run for $300$ time
steps, $51$ grid points, $\beta = 1$, and $\Delta t = 0.9 \Delta
t_{max}$, the results shown in figure \ref{fig:diffusion}.

\subsubsection{The Advection-Diffusion Equation}

By combining the two previous sections on advection and diffusion, the
advection-diffusion equation is:
\begin{equation}
  u_t = \beta u_xx - c u_x
\end{equation}
where the initial condition is advected with speed $c$ and diffusing
with rate $\beta$. The finite difference equation, stepping forward in
time, using the centred difference scheme for the second derivative
and backward for the first derivative, is:
\begin{equation}
  \frac{u_j^{n+1} - u_j^n}{\Delta t} = \beta \frac{u_{j+1}^n + 2 u_j^n
    + u_{j-1}^n}{\Delta x^2} - c \frac{u_j^n - u_{j-1}^n}{\Delta x}
\end{equation}

The von Neumann stability analysis is done similarly to before,
subsituting $u_j^n = G^n e^{i k j \Delta x}$ and solving for $G^1$:
\begin{equation}
  \begin{aligned}
    e^{i k j \Delta x} \frac{G^{n+1} - G^n}{\Delta t} &= \beta G^n
    \frac{e^{i k (j + 1) \Delta x} - 2 e^{i k j \Delta x} + e^{i k (j
        - 1) \Delta x}}{\Delta x^2} - c G^n \frac{e^{i k j \Delta x}
      e^{i k (j - 1) \Delta x}}{\Delta x} \\
    \frac{G^1 - 1}{\Delta t} &= \beta \frac{e^{i k \Delta x} - 2 +
      e^{- i k \Delta x}}{\Delta x^2} - c \frac{1 - e^{- i k \Delta
        x}}{\Delta x} \\
    \therefore G^1 &= 1 + 2 s \left( \cos(k \Delta x) - 1 \right) + r
    \left(e^{-i k \Delta x} - 1 \right)
  \end{aligned}
\end{equation}
Looking at the extreme values of $G^1$ and imposing the stability
condition $|G^1| \leq 1$, the upper extreme when $\cos(k \Delta x) =
1$ results in $|G^1| = 1$. On the lower extreme, when $\cos(k \Delta x
= -1$, the condition is:
\begin{equation}
  \begin{aligned}
    -1 &\leq 1 - 4s - 2r =1 - 2 \Delta t \left( \frac{2 \beta}{\Delta
        x^2} + \frac{2 c}{\Delta x} \right) \\
    \therefore \Delta t &\leq \Delta t_{max} = \frac{\Delta x^2}{c
      \Delta x + 2 \beta}
  \end{aligned}
\end{equation}

\end{document}
\message{ !name(austin_report.tex) !offset(-180) }
