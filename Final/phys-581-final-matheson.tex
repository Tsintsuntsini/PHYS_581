\documentclass[twocolumn]{article}
\usepackage{graphicx}
\usepackage{amsmath}
\usepackage{float}
\usepackage{amssymb} %Use of therefore symbol
\usepackage{hyperref}
\usepackage{caption}

\begin{document}
\title{PHYS 581 Final}
\author{Alex Matheson}
%\affiliation{Department of Physics and Astronomy, University of Calgary, Calgary AB T2N 1N4 Canada}
\date{\today}
\maketitle

\section{Introduction}
PHYS 581 was taught in 3 sections: Monte Carlo methods, Fourier analysis, and finite differences. In this final, all three areas will be examined using code and skills developed over the course of the class. 

\section{Methods}

\section{Conclusion}
Of the three methods considered in this final, both the Metropolis algorithm of Monte Carlo and the fast Fourier transform of Fourier analysis were included on a list of the most important algorithms of the 20th century. While finite differences were not included on that list due to their more limited scope outside of physics, their applicability to any system describable by differential equations (which includes much of modern mathematics and physics) makes them especially important when describing changing systems. In addition to their rigor, these systems excel at producing excellent results quickly, whether that be through a random process, an elegant re-ordering of frequency transformations, or through relaxation techniques. The tools covered in this course provide physicists with the ability to analyse large, complex sets of data efficiently and should find ample application in student's futures.

\begin{thebibliography}{00}
	\bibitem{ouyed}
	Ouyed and Dobler, PHYS 581 course notes, Department of Physics and Astrophysics, University of Calgary (2016).
	\bibitem{NR}
	W. Press et al., \emph{Numerical Recipes} (Cambridge University Press, 2010) 2nd. Ed.
\end{thebibliography}

\section{Appendix}
For access to the source codes used in this project, please visit \url{https://github.com/Tsintsuntsini/PHYS_581} for a list of files and times of most recent update.

\end{document}