\documentclass[twocolumn]{article}
\usepackage{graphicx}
\usepackage{amsmath}
\usepackage{float}
\usepackage{amssymb} %Use of therefore symbol
\usepackage{hyperref}
\usepackage{caption}



\begin{document}
\title{Lab 1: Monte Carlo Methods}
\author{Alex Matheson, Austin Nhung}
%\affiliation{Department of Physics and Astronomy, University of Calgary, Calgary AB T2N 1N4 Canada}
\date{\today}
\maketitle

\section{Introduction}

\section{Methods}
Fourier series are defined by calculating the fourier coefficients $a_n$ and $b_n$. These coefficients may be replaced when in a complex fourier series using a term $c_n$. Using the following equations:
\begin{equation}
\begin{split}
a_n =& c_n + c_{-n} \\
b_n =& i(c_n - c_{-n}) \\
c_n =& \frac{1}{2}(a_n - ib_n)
\end{split}
\end{equation}
In fourier series, the $a_n$ and $b_n$ correspond to even and odd 'components' of the function. In the case of an even function:
\begin{equation}
\begin{split}
a_n =& c_n + c_{-n} \\
b_n =& 0 \\
c_n =& \frac{1}{2}(a_n)
\end{split}
\end{equation}

And for odd functions:
\begin{equation}
\begin{split}
a_n =& 0 \\
b_n =& i(c_n - c_{-n}) \\
c_n =& \frac{-ib_n}{2}
\end{split}
\end{equation} 
 
It may be shown in both of the above series that the $a_n$ term for even functions and $b_n$ for odd functions will be proportional to the $c_n$ terms. 
\section{Conclusion}



\begin{thebibliography}{00}
	\bibitem{ouyed}
	Ouyed and Dobler, PHYS 581 course notes, Department of Physics and Astrophysics, University of Calgary (2016).
	\bibitem{NR}
	W. Press et al., \emph{Numerical Recipes} (Cambridge University Press, 2010) 2nd. Ed.
	\bibitem{Code}
	C. Hass and J. Burniston, MCMC Hill Climbing. Jupyter notebook, 2018.
\end{thebibliography}

\section{Appendix}
For access to the source codes used in this project, please visit \url{https://github.com/Tsintsuntsini/PHYS_581} for a list of files and times of most recent update.
	
\end{document}