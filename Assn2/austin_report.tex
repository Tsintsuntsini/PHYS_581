\documentclass[11pt, letterpage]{article}
\usepackage{amsmath}
\usepackage{graphicx}

\begin{document}

\section{Selected Topics}
\subsection{Power Spectrum and Filtering}
To investigate the properties of the power spectrum and the effects of
filtering, we consider the simple discretely sampled sinusoidal signal:
\begin{equation}
  V_j = A \sin(2 \pi f t_j - \phi)
  \label{eq:sine}
\end{equation}

%%%%%%%%%%%%%%%%%%%%%%%%%%%%%%%%%%%%% 
% SHOW POWER IS CONFINED TO ONE BIN
%%%%%%%%%%%%%%%%%%%%%%%%%%%%%%%%%%%%%

\subsubsection{Power Spectra}
The power spectrum of a continuous signal is simply the square of the continuous
Fourier transform. However, when computing the power spectrum of a discretely
sampled signal, the leakage of power from some frequencies to others is a
problematic result of the digitization of the signal. To illustrate this, the
power spectra for two slightly different frequencies are considered: $f = 60$Hz
and $f = 59.673$Hz.

\begin{figure}
  \includegraphics[width=\linewidth]{power_spectrum.png}
  \caption{
    The power spectra of equation \ref{eq:sine} with two very close frequencies.
  }
  \label{}
\end{figure}

\subsubsection{Windows}
One of the standard ways to reduce spectral leakage is to multiply the data in
time space with a window before performing the Fourier transform. Two of the
most common windows we look into are the Hann window and the Blackman-Harris
window. The Hann function and its Fourier transform are shown in figure
\ref{fig:hann_window}.

\begin{figure}
  \includegraphics[width=\linewidth]{hann_window.png}
  \caption{
    The Hann window (top) and its Fourier transform (bottom). It can be thought
    of as a portion of a cosine wave.
  }
  \label{fig:hann_window}
\end{figure}

Now the Hann window can be applied to the two signals that were used previously
with frequencies $60$Hz and $59.673$Hz, shown in figures \ref{fig:hann_0} and
\ref{fig:hann_1}, respectively. 

\begin{figure}
  \includegraphics[width=\linewidth]{hann_0.png}
  \caption{
    The Hann window applied to equation \ref{eq:sine} with $f = 60$Hz in time
    space (top) and in frequency space (bottom).
  }
  \label{fig:hann_0}
\end{figure}

\begin{figure}
  \includegraphics[width=\linewidth]{hann_1.png}
  \caption{
    The Hann window applied to equation \ref{eq:sine} with $f = 59.673$Hz in
    time space (top) and in frequency space (bottom).
  }
  \label{fig:hann_1}
\end{figure}

\begin{figure*}
  \includegraphics[\width=\pagewidth]{blackmann_harris.png}
  \caption{
    The Blackman-Harris window is also applied to equation \ref{eq:sin} with
    frequencies $f = 60$Hz and $f = 69.673$Hz. The functions in time space are
    shown in the top row while the bottom row shows the Fourier transform of the
    Blackman-Harris window and the power spectra of the functions above.
  }
  \label{fig:blackman_harris}
\end{figure*}

\section{General Applications}

\subsection{Heart Beats}
The data gives a time sequence of heart beats sampled at $125$Hz. Since the data
is evenly sampled and the number of samples is a power of two, padding is not
necessary in this case. The amplitude spectrum is shown in figure
\ref{fig:beats}. The two most dominant frequencies are approximately $2$Hz and
$4$kHz, with corresponding periods $0.5$s and $0.25$s, respectively.

\begin{figure}
  \includegraphics[width=\linewidth]{beats.png}
  \caption{
    The amplitude spectrum of the patient's heart beats. The data was sampled at
    $125$Hz and has $4096$ samples.
  }
  \label{fig:beats}
\end{figure}

The average resting heart rate is about $60$ to $100$ beats per minute. However,
this translates to periods within a range of $1$s to $1.6$s, which are
significantly faster than that of the patient. This would suggest that the
patient was not at rest before the measurement or some other reason for a faster
heart rate.

\subsection{Financial Series}
For this section, we analyze the stock prices for 6 companies, shown in figure
\ref{fig:stocks_time}.

\begin{figure}
  \includegraphics[width=\linewidth]{stocks_time.png}
  \caption{
    The monthly stock prices of six companies from $2002$ through to mid-$2007$.
  }
  \label{fig:stocks_time}
\end{figure}

This data is then used to compute the continuously compounded returns $R_i$,
given by the relation:
\begin{equation}
  R_i = \ln\left( \frac{P_i}{P_{i-1}} \right)
  \label{ccr}
\end{equation}
which are shown in figure \ref{fig:stocks_returns}. The autocorrelation of the
data points are shown in \ref{fig:stocks_ac}.

\begin{figure*}
  \includegraphics[width=\pagewidth]{stocks_returns.png}
  \caption{
    The continuously compunded returns of six companies. The returns were
    computed from the data in figure \ref{fig:stocks_time} using equation
    \ref{ccr}.
  }
  \label{fig:stocks_returns}
\end{figure*}

\begin{figure*}
  \includegraphics[width=\pagewidth]{stocks_ac.png}
  \caption{
    The autocorrelation data of six companies.
  }
  \label{fig:stocks_ac}
\end{figure*}

\begin{figure*}
  \includegraphics[width=\pagewidth]{stocks_power_spectrum.png}
  \caption{
    The power spectra of the stock prices of six companies. The frequencies are
    in units of per month.
  }
  \label{fig:stocks_ps}
\end{figure*}



\end{document}