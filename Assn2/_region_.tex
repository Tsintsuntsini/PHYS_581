\message{ !name(austin_report.tex)}\documentclass[11pt, letterpage]{article}
\usepackage{amsmath}

\begin{document}

\message{ !name(austin_report.tex) !offset(-3) }


\section{Selected Topics}

\subsection{DFT vs. FFT}
This section investigates different aspects of the discrete Fourier transform
(DFT) and its inverse (IDFT). First, consider the discrete boxcar frequency
spectrum:
\begin{equation}
  X(\omega) = \left\{
  \begin{array}{ll}
    1, & \text{for} \; |\omega| \leq \frac{\pi}{3} \\
    0, & \text{otherwise}
  \end{array}
  \right.
\end{equation}
We can transform this back into time space using the inverse transform.

Both the forward and inverse Fourier transform can be thought of as a linear
transformation on a vector, and thus can be represented through matrix
multiplication, with each element corresponding to the equation
\begin{equation}
  \begin{aligned}
    F_{nm} &= e^{- 2 \pi i \frac{nm}{N}} \\
    F_{nm}^{-1} &= \frac{1}{N} e^{2 \pi i \frac{nm}{N}}
  \end{aligned}
  \label{fft_mat_el}
\end{equation}
where $N$ is the total number of points. To transform a $3$-point vector, the
forward transformation matrix is then the $3 \times 3$ matrix
\begin{equation}
  F =
  \begin{pmatrix}
    e^{- \frac{2 \pi}{3} i} & e^{- \frac{4 \pi}{3} i} & e^{- 2 \pi i} \\
    e^{- \frac{4 \pi}{3} i} & e^{- \frac{8 \pi}{3} i} & e^{- 4 \pi i} \\
    e^{- 2 \pi i} & e^{- 4 \pi i} & e^{- 6 \pi i}
  \end{pmatrix}
\end{equation}
and its inverse is
\begin{equation}
  F^{-1} = \frac{1}{3}
  \begin{pmatrix}
    e^{\frac{2 \pi}{3} i} & e^{\frac{4 \pi}{3} i} & e^{2 \pi i} \\
    e^{\frac{4 \pi}{3} i} & e^{\frac{8 \pi}{3} i} & e^{4 \pi i} \\
    e^{2 \pi i} & e^{4 \pi i} & e^{6 \pi i}
  \end{pmatrix}
\end{equation}
We can easily confirm that these matrices are inverses of one another through
matrix multiplication so that their product is the identity matrix:
\begin{equation}
  F_{nm}F_{nm}^{-1}
  = \begin{pmatrix}
    1 & 0 & 0 \\
    0 & 1 & 0 \\
    0 & 0 & 1
  \end{pmatrix}
\end{equation}
We can also prove this by applying the forward and inverse transformation to a
simple vector such as $x = \begin{bmatrix} 1 & 0 & -1 & 0 \end{bmatrix}^T$:
\begin{equation}
  \begin{aligned}
    \mathbb{F} \left\{ x \right\}
    = F_{nm}
  \end{aligned}
\end{equation}


\end{document}
\message{ !name(austin_report.tex) !offset(-75) }
