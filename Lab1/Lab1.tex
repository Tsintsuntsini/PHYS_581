\documentclass[twocolumn]{article}
\usepackage{graphicx}
\usepackage{amsmath}
\usepackage{amssymb} %Use of therefore symbol
\usepackage{hyperref}
\usepackage{caption}



\begin{document}
\title{Lab 1: Monte Carlo Methods}
\author{Alex Matheson, Austin Nguyen}
%\affiliation{Department of Physics and Astronomy, University of Calgary, Calgary AB T2N 1N4 Canada}
\date{\today}
\maketitle

\section{Introduction}

\section{Methods}
\subsection{Random Numbers}


\subsection{Light Diffusion}
A number of equations were provided for different parameters in a light diffusion scenario. In this scenario, a photon enters a uniform slab and interacts with matter inside. The possible interactions are absorption, scattering, or exiting the medium. To simulate the path of a single photon through the slab, each parameter in the equations need to be randomly sampled. These parameters are not necessarily uniform. First, a sampling equation for optical depth was determined:
\begin{equation}
\begin{split}
P(\tau) d\tau =& e^{-\tau} d\tau \\
\therefore F_{\tau}(\tau) =& e^{-\tau} \\
\tau =& F^{-1}_{\tau}(u) \\
\tau =& e^{u}
\end{split}
\end{equation}

Next, the distribution for initial orientation $\theta$ was provided. From this random samples could be determined:
\begin{equation}
\begin{split}
P(\theta) d\theta =& \frac{1}{2} \sin(\theta) \\
\therefore F_{\theta}(\theta) =& \frac{1}{2} \sin(\theta)\\
\theta =& F^{-1}_{\theta}(u) \\
\theta =& \sin^{-1}(2u)
\end{split}
\end{equation}

Lastly, angle $\phi$ needed to be considered. Thankfully, this variable was already uniform.

The length traveled through the medium was dependent on the optical depth.
\begin{equation}
\begin{split}
L =& \int_{0}^{\tau_{max}}\sigma n dz \\
\end{split}
\end{equation}

For most of the path of a photon, the exit condition will not be in play. For the rest of the medium, the probability of a photon being scattered is:
\begin{equation}
prob = \frac{P_s}{P_s + P_a}
\end{equation}

Since the two probabilities must add together to $1$, the absorb or scatter condition has value $1$ associated with complete scattering (when $P_s = 1$ and $P_a=0$) or value $0$ for complete absorption (when $P_s = 0$ and $P_a=1$).
\section{Discussion}


\section{Conclusion}

\begin{thebibliography}{00}
	\bibitem{ouyed}
	Ouyed and Dobler, PHYS 581 course notes, Department of Physics and Astrophysics, University of Calgary (2016).
	\bibitem{NR}
	W. Press et al., \emph{Numerical Recipes} (Cambridge University Press, 2010) 2nd. Ed.
\end{thebibliography}

\section{Appendix}

	
\end{document}