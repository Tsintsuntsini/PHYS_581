\subsection{Metropolis-Hastings}
\subsubsection{The Algorithm}
The Metropolis-Hastings algorithm is a variation of the Metropolis algorithm, so
to explore this idea, a simple Metropolis random-walk is simulated. The big idea
behind the Metropolis algorithm is creating ``walkers'' which travel through the
probability space, and the walkers approach the correct probability distribution
after a large number of steps. The main difference between Metropolis and
Metropolis-Hastings is the addition of a proposal distribution from which the
direction and size of each step is sampled from. The proposal distribution can
be symmetrical, as is in the case of Metropolis, or asymmetrical, as it is in
Metropolis-Hastings.

Each step is given a weight, defined as
\begin{equation}
  \alpha(x') = \min \left( 1, \frac{p(x') q(x' | x)}{p(x) q(x | x')} \right)
  \label{eq:acceptance}
\end{equation}
where $p$ is the target distribution, $q$ is the proposal distribution, and $x'$
is the proposed step from $x$. This weighting forces steps toward a local
maximum to always be accepted, but steps away from the local maximum are
accepted with some probability. The state-to-state process describes a Markov
chain, which was explored earlier. Another consequence of this weighting is that
neither the target distribution nor the proposal distribution need to be
normalized. This property can be very advantageous when sampling a complicated
distribution.

\subsubsection{A Simple Example}
For a simple example, we consider the target distribution
\begin{equation}
  P(x) = \frac{1}{2 \sqrt{2}} \left( \sin(5x) + \sin(2x) + 2 \right) e^{-x^2}
  \label{eq:example}
\end{equation}
and the proposal distribution
\begin{equation}
  q(x | x') = \frac{1}{\sqrt{2 \pi \sigma^2}} \exp \left( - \frac{(x' - x)^2}{2 \sigma^2} \right)
\end{equation}
Since the normal distribution is symmetrical, this is actually equivalent to
distribution which moves in the opposite direction, from $x'$ to $x$, since the
square term in the exponential can be reversed. Thus, for the normal
distribution,
\begin{equation}
  q(x | x') = q(x' | x)
\end{equation}

The acceptance probability from \ref{eq:acceptance} is then reduced to
\begin{equation}
  \alpha(x') = \min \left( 1, \frac{p(x')}{p(x)} \right)
\end{equation}
This returns the acceptance probability of the simple Metropolis algorithm. The
intuitive reason why the ratio of proposal distributions disapears is that the
ratio describes the relative probability of moving one way, from $x$ to $x'$,
versus the other. The reason this ratio is included is to give greater control
over how the walkers traverse the probability space. The motivation to seek
greater control is to avoid completely relying on the target distribution, where
the walkers may get trapped in a local maxima. The problem is more daunting as
the number of degrees of freedom increases, leading to the curse of
dimensionality.

Once the acceptance probability of a proposed step is calculated, a random
number $r$ from a uniform distribution is generated, and the step is accepted
with the probabilities
\begin{equation}
  x_{n+1} = \left\{
    \begin{array}{lr}
      x', & \text{if} \, r <= \alpha(x') \\
      x_n & \text{if} \, r > \alpha(x')
    \end{array}
  \right.
\end{equation}

Continuing on with the example distribution \ref{eq:example}, the
Metropolis-Hastings algorithm was implemented with three different values for
standard deviation $\sigma = 0.025, 1.0, 50$ in order to compare the effects of
using a very narrow or very wide proposal distribution.

% Code results

\subsubsection{The Burn-in Phase}
The burn-in phase refers to the process of the walkers moving toward an
equilibrium. The number of steps this takes is highly dependent on the choice of
starting values and how suitable the proposal distribution is. 

